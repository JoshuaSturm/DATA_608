\documentclass[]{article}
\usepackage{lmodern}
\usepackage{amssymb,amsmath}
\usepackage{ifxetex,ifluatex}
\usepackage{fixltx2e} % provides \textsubscript
\ifnum 0\ifxetex 1\fi\ifluatex 1\fi=0 % if pdftex
  \usepackage[T1]{fontenc}
  \usepackage[utf8]{inputenc}
\else % if luatex or xelatex
  \ifxetex
    \usepackage{mathspec}
  \else
    \usepackage{fontspec}
  \fi
  \defaultfontfeatures{Ligatures=TeX,Scale=MatchLowercase}
\fi
% use upquote if available, for straight quotes in verbatim environments
\IfFileExists{upquote.sty}{\usepackage{upquote}}{}
% use microtype if available
\IfFileExists{microtype.sty}{%
\usepackage{microtype}
\UseMicrotypeSet[protrusion]{basicmath} % disable protrusion for tt fonts
}{}
\usepackage[margin=1in]{geometry}
\usepackage{hyperref}
\hypersetup{unicode=true,
            pdftitle={DATA 608 - Final Project Proposal},
            pdfauthor={Joshua Sturm},
            pdfborder={0 0 0},
            breaklinks=true}
\urlstyle{same}  % don't use monospace font for urls
\usepackage{graphicx,grffile}
\makeatletter
\def\maxwidth{\ifdim\Gin@nat@width>\linewidth\linewidth\else\Gin@nat@width\fi}
\def\maxheight{\ifdim\Gin@nat@height>\textheight\textheight\else\Gin@nat@height\fi}
\makeatother
% Scale images if necessary, so that they will not overflow the page
% margins by default, and it is still possible to overwrite the defaults
% using explicit options in \includegraphics[width, height, ...]{}
\setkeys{Gin}{width=\maxwidth,height=\maxheight,keepaspectratio}
\IfFileExists{parskip.sty}{%
\usepackage{parskip}
}{% else
\setlength{\parindent}{0pt}
\setlength{\parskip}{6pt plus 2pt minus 1pt}
}
\setlength{\emergencystretch}{3em}  % prevent overfull lines
\providecommand{\tightlist}{%
  \setlength{\itemsep}{0pt}\setlength{\parskip}{0pt}}
\setcounter{secnumdepth}{0}
% Redefines (sub)paragraphs to behave more like sections
\ifx\paragraph\undefined\else
\let\oldparagraph\paragraph
\renewcommand{\paragraph}[1]{\oldparagraph{#1}\mbox{}}
\fi
\ifx\subparagraph\undefined\else
\let\oldsubparagraph\subparagraph
\renewcommand{\subparagraph}[1]{\oldsubparagraph{#1}\mbox{}}
\fi

%%% Use protect on footnotes to avoid problems with footnotes in titles
\let\rmarkdownfootnote\footnote%
\def\footnote{\protect\rmarkdownfootnote}

%%% Change title format to be more compact
\usepackage{titling}

% Create subtitle command for use in maketitle
\newcommand{\subtitle}[1]{
  \posttitle{
    \begin{center}\large#1\end{center}
    }
}

\setlength{\droptitle}{-2em}
  \title{DATA 608 - Final Project Proposal}
  \pretitle{\vspace{\droptitle}\centering\huge}
  \posttitle{\par}
  \author{Joshua Sturm}
  \preauthor{\centering\large\emph}
  \postauthor{\par}
  \predate{\centering\large\emph}
  \postdate{\par}
  \date{03/27/2018}


\begin{document}
\maketitle

\subsection{Mission Statement}\label{mission-statement}

My goal is to explore the prevalence of obesity amongst adults in the
United States, by county. I intend to find if there is any correlation
between being overweight and other factors, such as economical
(percentage of population living below the federal poverty level), and
educational (highest level of schooling completed).

According to the
\href{https://www.cdc.gov/obesity/adult/causes.html}{Centers for Disease
Control and Prevention}, the annual nationwide productive costs of
obesity obesity-related absenteeism range between \$3.38 billion (\$79
per obese individual) and \$6.38 billion (\$132 per obese individual).
There are incalculable economical impacts on the healthcare industry.
Many citizens are unable to join the army due to being unable to pass
the physical exams.

Clearly, obesity impacts not just the sufferers, but other Americans as
well. Population percentage with obesity rises each year. We need to
find better ways to prevent, treat, and rid obesity from our population.

\subsection{Data}\label{data}

I will source the data from several different datasets.

\begin{itemize}
\tightlist
\item
  \href{https://factfinder.census.gov/faces/nav/jsf/pages/index.xhtml}{American
  FactFinder}
\item
  \href{https://www.cdc.gov/diabetes/data/countydata/countydataindicators.html}{Centers
  for Disease Control and Prevention}
\end{itemize}

\subsection{Tools}\label{tools}

I haven't yet decided how I will present this project. I would like to
make it interactive, e.g.~a Shiny app, but I don't yet know if it's
feasible, given the number of variables. As I explore the data more, I
will finalize my plan.


\end{document}
