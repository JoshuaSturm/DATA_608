\documentclass[]{article}
\usepackage{lmodern}
\usepackage{amssymb,amsmath}
\usepackage{ifxetex,ifluatex}
\usepackage{fixltx2e} % provides \textsubscript
\ifnum 0\ifxetex 1\fi\ifluatex 1\fi=0 % if pdftex
  \usepackage[T1]{fontenc}
  \usepackage[utf8]{inputenc}
\else % if luatex or xelatex
  \ifxetex
    \usepackage{mathspec}
  \else
    \usepackage{fontspec}
  \fi
  \defaultfontfeatures{Ligatures=TeX,Scale=MatchLowercase}
\fi
% use upquote if available, for straight quotes in verbatim environments
\IfFileExists{upquote.sty}{\usepackage{upquote}}{}
% use microtype if available
\IfFileExists{microtype.sty}{%
\usepackage{microtype}
\UseMicrotypeSet[protrusion]{basicmath} % disable protrusion for tt fonts
}{}
\usepackage[margin=1in]{geometry}
\usepackage{hyperref}
\PassOptionsToPackage{usenames,dvipsnames}{color} % color is loaded by hyperref
\hypersetup{unicode=true,
            pdftitle={DATA\_608\_Final\_Project},
            pdfauthor={Joshua Sturm},
            colorlinks=true,
            linkcolor=Maroon,
            citecolor=Blue,
            urlcolor=blue,
            breaklinks=true}
\urlstyle{same}  % don't use monospace font for urls
\usepackage{graphicx,grffile}
\makeatletter
\def\maxwidth{\ifdim\Gin@nat@width>\linewidth\linewidth\else\Gin@nat@width\fi}
\def\maxheight{\ifdim\Gin@nat@height>\textheight\textheight\else\Gin@nat@height\fi}
\makeatother
% Scale images if necessary, so that they will not overflow the page
% margins by default, and it is still possible to overwrite the defaults
% using explicit options in \includegraphics[width, height, ...]{}
\setkeys{Gin}{width=\maxwidth,height=\maxheight,keepaspectratio}
\IfFileExists{parskip.sty}{%
\usepackage{parskip}
}{% else
\setlength{\parindent}{0pt}
\setlength{\parskip}{6pt plus 2pt minus 1pt}
}
\setlength{\emergencystretch}{3em}  % prevent overfull lines
\providecommand{\tightlist}{%
  \setlength{\itemsep}{0pt}\setlength{\parskip}{0pt}}
\setcounter{secnumdepth}{0}
% Redefines (sub)paragraphs to behave more like sections
\ifx\paragraph\undefined\else
\let\oldparagraph\paragraph
\renewcommand{\paragraph}[1]{\oldparagraph{#1}\mbox{}}
\fi
\ifx\subparagraph\undefined\else
\let\oldsubparagraph\subparagraph
\renewcommand{\subparagraph}[1]{\oldsubparagraph{#1}\mbox{}}
\fi

%%% Use protect on footnotes to avoid problems with footnotes in titles
\let\rmarkdownfootnote\footnote%
\def\footnote{\protect\rmarkdownfootnote}

%%% Change title format to be more compact
\usepackage{titling}

% Create subtitle command for use in maketitle
\newcommand{\subtitle}[1]{
  \posttitle{
    \begin{center}\large#1\end{center}
    }
}

\setlength{\droptitle}{-2em}
  \title{DATA\_608\_Final\_Project}
  \pretitle{\vspace{\droptitle}\centering\huge}
  \posttitle{\par}
  \author{Joshua Sturm}
  \preauthor{\centering\large\emph}
  \postauthor{\par}
  \predate{\centering\large\emph}
  \postdate{\par}
  \date{May 1, 2018}


\begin{document}
\maketitle

As can be seen in the source files, I originally used multiple files to
get all the information I wanted, before finding a
\href{https://factfinder.census.gov/faces/tableservices/jsf/pages/productview.xhtml?pid=ACS_16_1YR_S1703\&prodType=table}{different
source} that had all of it contained in a single table. I included the
years 2010 through 2016.

Due to storage and memory constraints, I had to limit which columns I
could use. I tried to choose those which I believed to provide the
greatest insight. I did some basic tidying in Excel (such as removing /
renaming columns) before loading the file to work with in javascript.

Since my only knowledge of javascript comes from this course, the next
part took the bulk of my time. I first tried working directly with my
data, before learning that I had to convert each column. I tried for
hours to loop through all columns to convert automatically, but when
nothing worked, I gave up and did each variable manually.

Once I finally got the file loaded with d3, I ran into several problems
with crossfilter. After toiling for \emph{days}, I learned that
crossfilter only accepts ``long'' data, and my file was wide. Through
StackOverflow, I learned of something called melt.js, and finally I was
able to get crossfilter working - for a single variable. I still haven't
figured out how to filter a column in d3/crossfilter - or if it's even
possible.

Finally, I resorted to importing my file into R, and melting the entire
file\ldots{}multiple times. My reasoning was that I wanted each variable
to have its own column and corresponding ``count'' column. There is most
likely a better way to do this, but I haven't yet found one. The
consequence of doing it this way was that each variable was repeated
numerous times, and so the crossfilter output was really a multiple of
the true values. Using R and Excel, I found out what that multiple was
(it was different for each column), and divided by that. The numbers
were now close to the real values, although they were in decimal format.
When trying to round to an integer, I was getting large errors, which I
assume is due to floating point arithmetic error.

The end result is that the dashboard\ldots{}works. The file is quite
large, so it takes \textasciitilde{}10 seconds to load, but once it's
committed to memory, crossfilter works \emph{really} quickly; the
calculations and transitions are near instantaneous. If I had more time,
I would further enhance the display with CSS, but I fixed the position
of a few graphs to make sure none of them overlap.

I could have made a really nice app in Plotly or Shiny in a fraction of
the time (I spent 60+ hours on this project), but I wanted to take on
this challenge. Upon completion of this project, and with a working
demo, I am glad that I took the more difficult route. I may have missed
a few deadlines because of it, but I learned so much from this struggle,
that it made it all worthwhile.

Thank you, Professor Ferrari, for an amazing and educational semester!


\end{document}
